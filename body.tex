
\section{Introduction}
As outlined in \citeds{DMTN-199} only the Rubin team have access to pixel data for an embargo period of 80 hours and thirty days during commissioning. We assume this also means PNG or other format images captured from screens etc which may show pixel data e.g. from RubinTV on the summit.

\section{General policy} \label{sec:genpol}
Many scenarios are discussed in \citeds{SITCOMTN-076} concerning information sharing in commissioning.
In general we may state:

No images seen by the team in any communication space should be shared outside the space at anytime.
Images should not be left open on laptops at meetings or public spaces.
Images should not be shown in in presentations.


This is even more important before first light where we need to carefully control the public first light images.

\section {Systems that need policies}
In various discussions the list of tools we need to have polices for are:

\begin{enumerate}
    \item Confluence
    \item Jira
    \item github (e.g., draft technotes, Construction paper drafts, notebooks with plots)
    \item technotes on lsst.io
    \item Slack
    \item summit tooling (i.e RubinTV)
    \item USDF tooling
    \item email lists
\end{enumerate}

We discuss each in a section below.
Tech notes and Slack are possibly most heavily used so lets start there.
\subsection{Slack}
We have a wide variety of very open channels on Slack, these include large numbers of science collaborators and member of the astronomy community.
So far with AuxTel and with Simonyi with star tracker images have been shared frequently on certain slack channels which are not private.
There is the possibility to have private channels on Slack - we have few and tend to avoid them for several reasons, they are not discoverable some bots don't work in them etc.
Even using Private channels we would face a problem of members potentially inviting non staff to the channel where images may be shown.
Much as we dislike having ANOTHER Slack space resurrecting the old LSST slack space adding only the team members with image access to it and using it for all the the nighttime summit channels would be a clean solution.
In the space all channels could be public and no retracting with posting images.
We would have to decide if summit announce moved or not.

Whether we make private channels or switch to a private space - this needs to be done preferably before July 2024.
\subsubsection{Proposed policy}
Resurrect or create an new Slack org for restricted use of the Rubin team, at least during commissioning.
We could revisit after commissioning.
See also \secref{sec:genpol}.

Alternatively - do not post pixel images on Slack - only use links to secured sources.
Would people support this ? We could allow screen grabs of Ghosts or other artifacts esp as they would have no location information.

\subsection{technotes on lsst.io}
The lsst.io site is intentionally public.
Some notes could get auth protection for some time - not sure of feasibility by end 2024.

If we can not make lsst.io authenticated - private repos with PDF in the repo may be alternative (this was done for DMTN-199 when it was private)


\subsection{Github }
There may be many things in repos in github e.g., draft technotes, Construction paper drafts, notebooks with plots.
These need to be in private repos.
Even so private repos are visible to all team members on github so care needs to be taken.

An alternative would be again use authenticated links to images until they are not embargoed and keep all notes public.

\subsubsection{Proposed policy for Github}
Rendered notebooks containing image data must not appear in public repos.
Notebooks, papers etc with pixel data should only appear in \emph{private SITCOM} repos.
The github SITCOM team members need to be continuously and carefully aligned with actual commissioning team membership.

\subsection{Confluence}
Much of confluence is public (all of DM is public by choice).
SITCOM is restricted to login but not more than that.
The main area where one may expect to see images would be performance analysis - an admittedly  cursory glance at all attachments suggests there are not many.
There are many links to notebooks - this is fine as notebooks require execution/access.

\subsubsection{Proposed policy for Confluence}
Do not up load any pixel images or parts of pixel images or screenshots of images to confluence.
Use an authenticated link to any image rather than the actual image if it is needed in a page.


\subsection{Jira}
As for confluence all DM tickets are public in Jira.
Other projects are password protected but anyone with access to Jira may view such tickets.
Its not obvious we could even secure such a system.

\subsubsection{Proposed policy for Jira}
Do not up load any pixel images or parts of pixel images to Jira.
Use an authenticated link to any image rather than the actual image if it is needed in a ticket.
It is acceptable to put a screenshot of an effect on a CCD if it is pertinent to the issue.
Such a screen grab should be no more than a quarter of  one full CCD image in size and the total images on a ticket should not accumulate to more than one CCD in size.

\subsection{summit tooling (i.e RubinTV)}
This all has to be secured for commissioning use.
We should aim to put this in place around ComCam on sky July 2024.

\subsubsection{Proposed policy for summit tooling}
Summit tooling needs to go behind the Summit 2FA VPN which is in place as of March 6.
RubinTV needs to have authentication added and be restricted to the same Summit IPA groups.

\subsection{USDF tooling}
USDF tooling should only be accessible to USDF account holders who are in the commissioning or operations teams.
\subsubsection{Proposed policy for USDF tooling}
Ensure only account holders in allowed groups have access to USDF tooling.
Allowed groups are those involved in commissioning plus DM and SP staff involved in QA and essential operations trouble shooting at USDF.

\subsection{Emails and email  lists }
Email is inherently insecure (few of us use encryption) and list servers are fairly open.

\subsubsection{Proposed policy for Email}
Do not attach Rubin pixel images or parts of image to emails.
Use authenticated links to images where needed.


