\documentclass[DM,lsstdraft,authoryear,toc]{lsstdoc}
\input{meta}

% Package imports go here.

% Local commands go here.

%If you want glossaries
%\input{aglossary.tex}
%\makeglossaries

\title{Data security for Rubin communication channels}

% Optional subtitle
% \setDocSubtitle{A subtitle}

\author{%
William O'Mullane
}

\setDocRef{DMTN-286}
\setDocUpstreamLocation{\url{https://github.com/lsst-dm/dmtn-286}}

\date{\vcsDate}

% Optional: name of the document's curator
% \setDocCurator{The Curator of this Document}

\setDocAbstract{%
We have a lot of fairly open communication channels which should not receive image data during embargo periods. This is more of a dilemma  in commissioning with the longer embargo period but persists throughout operations. This note discusses the issue and draws some conclusions.
}

% Change history defined here.
% Order: oldest first.
% Fields: VERSION, DATE, DESCRIPTION, OWNER NAME.
% See LPM-51 for version number policy.
\setDocChangeRecord{%
  \addtohist{1}{2024-03-06}{Unreleased.}{William O'Mullane}
}


\begin{document}

\mkshorttitle



\section{Introduction}
As outlined in \citeds{DMTN-199} only the Rubin team have access to pixel data for an embargo period of 80 hours and thirty days during commissioning. We assume this also means PNG or other format images captured from screens etc which may show pixel data e.g. from RubinTV on the summit.

\section{General policy} \label{sec:genpol}
Many scenarios are discussed in \citeds{SITCOMTN-076} concerning information sharing in commissioning.
In general we may state:

No images seen by the team in any communication space should be shared outside the space at anytime.
Images should not be left open on laptops at meetings or public spaces.
Images should not be shown in in presentations.


This is even more important before first light where we need to carefully control the public first light images.

\section {Systems that need policies}
In various discussions the list of tools we need to have polices for are:

\begin{enumerate}
    \item Confluence
    \item Jira
    \item github (e.g., draft technotes, Construction paper drafts, notebooks with plots)
    \item technotes on lsst.io
    \item Slack
    \item summit tooling (i.e RubinTV)
    \item USDF tooling
    \item email lists
\end{enumerate}

We discuss each in a section below.
Tech notes and Slack are possibly most heavily used so lets start there.
\subsection{Slack}
We have a wide variety of very open channels on Slack, these include large numbers of science collaborators and member of the astronomy community.
So far with AuxTel and with Simonyi with star tracker images have been shared frequently on certain slack channels which are not private.
There is the possibility to have private channels on Slack - we have few and tend to avoid them for several reasons, they are not discoverable some bots don't work in them etc.
Even using Private channels we would face a problem of members potentially inviting non staff to the channel where images may be shown.
Much as we dislike having ANOTHER Slack space resurrecting the old LSST slack space adding only the team members with image access to it and using it for all the the nighttime summit channels would be a clean solution.
In the space all channels could be public and no retracting with posting images.
We would have to decide if summit announce moved or not.

Whether we make private channels or switch to a private space - this needs to be done preferably before July 2024.
\subsubsection{Proposed policy}
Resurrect or create an new Slack org for restricted use of the Rubin team, at least during commissioning.
We could revisit after commissioning.
See also \secref{sec:genpol}.

\subsection{technotes on lsst.io}

\subsection{github }
(e.g., draft technotes, Construction paper drafts, notebooks with plots)

\subsection{Confluence}
Much of confluence is public (all of DM is public by choice).
SITCOM is restricted to login but not more than that.
The main area where one may expect to see images would be performance analysis - an admittedly  cursory glance at all attachments suggests there are not many.
There are many links to notebooks - this is fine as notebooks require execution/access.

\subsubsection{Proposed policy}
Do not up load any pixel images or parts of pixel images or screenshots of images to confluence.
Use an authenticated link to any image rather than the actual image if it is needed in a page.



\subsection{Jira}
\subsection{summit tooling (i.e RubinTV)}
\subsubsection{Proposed policy}
Summit tooling needs to go behind the Summit 2FA VPN which is in place as of March 6.
RubinTV needs to have authentication added and be restricted to the same Summit IPA groups.

\subsection{USDF tooling}
\subsection{email lists }


\appendix
% Include all the relevant bib files.
% https://lsst-texmf.lsst.io/lsstdoc.html#bibliographies
\section{References} \label{sec:bib}
\renewcommand{\refname}{} % Suppress default Bibliography section
\bibliography{local,lsst,lsst-dm,refs_ads,refs,books}

% Make sure lsst-texmf/bin/generateAcronyms.py is in your path
\section{Acronyms} \label{sec:acronyms}
\addtocounter{table}{-1}
\begin{longtable}{p{0.145\textwidth}p{0.8\textwidth}}\hline
\textbf{Acronym} & \textbf{Description}  \\\hline

DM & Data Management \\\hline
DMTN & DM Technical Note \\\hline
IPA & FreeIPA - Identity, Policy, Audit \\\hline
LSST & Legacy Survey of Space and Time (formerly Large Synoptic Survey Telescope) \\\hline
PNG & Portable Network Graphics \\\hline
SITCOM & System Integration, Test and Commissioning \\\hline
USDF & United States Data Facility \\\hline
VPN & virtual private network \\\hline
\end{longtable}

% If you want glossary uncomment below -- comment out the two lines above
%\printglossaries





\end{document}
