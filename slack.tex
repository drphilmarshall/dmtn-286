\subsection{Slack} \label{sec:slack}

We have a wide variety of very open channels on Slack, these include large numbers of science collaborators and member of the astronomy community.
So far, AuxTel and Simonyi star tracker images have been shared frequently on certain Slack channels which are not private.
There is the possibility to have private channels on Slack --- we have few and tend to avoid them for several reasons: they are not discoverable, some bots don't work in them, etc.
Even using Private channels, we would face a problem of members potentially inviting non-staff to the channel where images may be shown.


The current (LSSTC) Slack workspace was intentionally conceived as a community (including LSST Science Collaborations) inclusive space. This is at odds with a number of goals:

\begin{itemize}
\item  All channels would be available to all staff in the space --- no problem sharing images as long as ALL agree they never leave the space.
This allows us a great project space to enjoy successes as they arise.
\item  Fail quietly: We need a space where we can openly discuss problems (e.g. “omg what if we crack the AuxTel array” discussion in an open channel)
\item Embargo: Freely post data without having to stop and think whether it is embargoed.
\item GitOps: Slack is “the UNIX command line” for highly distributed teams. SQuaRE offers (and plans on expanding its offerings) of slackbots that actively manipulate project resources, report on project data, issue alerts on operational services etc. Being assured that only trusted staff have access allows us to expand what these services can do.
\item High Priority:  This gives a high-importance lower volume workspace that we can less disruptively monitor out of hours, during holidays and vacations and overall busy times, reducing the communication attack surface.
\item Privacy: We already pin phone numbers to certain channels, and there are concerns about freely sharing staff phone numbers, vacation schedules etc
\item On/Off-boarding: Even if someone is off-boarded from the project, there are legitimate reasons they should still maintain their wider community slack access. A staff-only slack can be tightly controlled together with other high value project access.
\item Slack Culture: We have issues where slack cultures class, eg. community folks at-channel just because a seminar is about to start, respecting quiet days etc. A separate workspace can maintain a more ops-oriented slack culture. It is also easier to respond to inappropriate behavior when it’s your own staff engaging in it.
\end{itemize}

There are some reasons we may not with to have a private Slack:

\begin{itemize}
\item ``We are at risk of abandoning the community'': The motivation behind LSSTC was to establish good working relationships with non-project staff, particularly DESC. This was at a time when Slack had very poor support for multiple workspaces and guest channel access. Moreover the LSSTC workspace is now well established for this purpose and ops leaders will remain reachable on it.
\item  ``It's fine, we have private channels'': The proliferation of private channels have been a bit of a nightmare. We're always wondering who should or should not have access to them, forget to add new people, forget to off-board departing people, are unclear what channels  should or should not be private, etc. It also has led to reduced transparency within the project --- a number of channels that were open in the old slack space were made private when we moved to the LSSTC slack.
\item  ``Too hard to determine out who is staff'': It's true that the proliferation of in-kind, grad students and other participants have muddied the waters. However this is a problem we still have in the current set-up -  it's just less obvious. The most clear heuristics include “are we paying (corollary: can we “fire”) someone who has violated project rules; are they on the builder's accrual list; do they have summit access; etc.
\item  ``We already say we don't do community user support on Slack'': This is true, we do say it, but we should recognize that it's emotionally hard work when someone is asking a question on Slack to determine whether they are staff or no, and if not to tell them to go elsewhere. Sometimes we just answer without checking, which muddies our stance.
\item  ``We're too busy for this kind of change'': True, but this will only get worse. There is also a plan being prepared to provide more consistent naming for certain types of channels (support, status, etc.) and update default status semantics,  so this would be a good time to implement that.
\end{itemize}


Much as we dislike having ANOTHER Slack space resurrecting the old LSST Slack space adding only the team members with image access to it and using it for all the the nighttime summit channels would be a clean solution.
Whether we make private channels or switch to a private space --- this needs to be done preferably before July 2024.
We need to start a concrete plan for this.

\subsubsection{Proposed policy}
See also \secref{sec:genpol}.
Resurrect lsst.slack.com Slack org for restricted use of the Rubin team, at least during commissioning.

Reuse of lsst.slack.com as some advantages:

\begin{itemize}
\item  This is already set up grandfathered as free so we don't risk applying and being turned down for another free workspace
\item  Can deliberately blur construction / commissioning / ops lines since it is free
\item  Rubin would be in total control of configuration, access and any paid features
\item  Workspace is already configured for our use which would speed up any transition

\end{itemize}
We could revisit after commissioning.
